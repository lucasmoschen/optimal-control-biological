Nesse capítulo serão realizados os laboratórios 1, 2 e 3. Os laboratórios
tratam de aplicações simplificadas de situações reais que envolvem conceitos
de biologia. Eles são auto-contidos se a teoria dos capítulos
anteriores for conhecida. 

\begin{enumerate}[label=\textbf{Lab \arabic*:}]
    \item Exemplo Introdutório
    
    O algoritmo apresentado no capítulo \ref{ch:4} é apresentado na linguagem
    Python e cada passo é explicado. Um simples exemplo do problema de
    controle ótimo é desenvolvido. 

    \item Mofo e Fungicida 
    
    Modelo simplificado do crescimento de um mofo contra a ação de um
    fungicida, que serve de controle para o aumento do mofo, dado que este é
    um efeito indesejado e deve, portanto, ser minimizado. 

    \item Bactéria 
    
    Uma bacteria tem seu crescimento acelerado por um químico, mas
    simultaneamente é criado um subproduto tóxico a ela. Queremos que no
    final do experimento, tenhamos o máximo dessa bactéria, mas sem usar muito
    químico. 
\end{enumerate}